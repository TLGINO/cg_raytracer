\documentclass{article}

\usepackage{amsmath}
\usepackage{amssymb}
\usepackage{gensymb}
\usepackage{multirow}

\title{CG assignment 4}
\author{Jeferson Morales Mariciano, Martin Lettry}
\date{\today}

\begin{document}

\maketitle

\section*{Ex 1 [5 points]}
Let us consider a 2D coordinate system with points $p_1 = [1, 1]^\intercal$,
and $p = [1.5, 2.5]^\intercal$.
Additionally, let's define a vector $u = p - p_1$.
Perform the following tasks:

\subsection*{Task 1 [1 points]}
Construct two matrices, $R_{90}$, $T \in \mathbb{R}^{3 \times 3}$,
such that the first one performs counter-clockwise rotation around
the center of the coordinates system by angle $90 \degree$, and the second one performs translation
by vector $t = [1, -2]^\intercal$, in homogeneous coordinates.

\[
    p =
    \begin{bmatrix}
        1.5 \\ 2.5 \\
    \end{bmatrix}
    , p_1 =
    \begin{bmatrix}
        1 \\ 1 \\
    \end{bmatrix}
\]

\[
    u = p - p_1 =
    \begin{bmatrix}
        0.5 \\ 1.5 \\
    \end{bmatrix}
\]
From generic counter clockwise rotation in 2D coord system:
\[
    R(\alpha) =
    \begin{bmatrix}
        cos(\alpha) & -sin(\alpha) \\
        sin(\alpha) & cos(\alpha)  \\
    \end{bmatrix}
\]
Then, rotation matrix with homogeneous coordinates for $\alpha = 90\degree$ is:
\[
    R_{90} =
    \begin{bmatrix}
        cos(90) & -sin(90) & 0 \\
        sin(90) & cos(90)  & 0 \\
        0       & 0        & 1 \\
    \end{bmatrix}
    =
    \begin{bmatrix}
        0 & -1 & 0 \\
        1 & 0  & 0 \\
        0 & 0  & 1 \\
    \end{bmatrix}
\]
Given rotation matrix $R$ and translation vector $t$ for 2D space coordinates,
the affine transformation matrix $T$ is:
\begin{equation*}
    \begin{aligned}
        T & =
        \begin{bmatrix}
            \multicolumn{3}{c}{\multirow{2}{*}{R}} & \multirow{2}{*}{t}         \\
            \multicolumn{4}{c}{}                                                \\
            0                                      & \dots              & 0 & 1 \\
        \end{bmatrix}
        \in \mathbb{R}^{3 \times 3} \\[-0.5ex]
          & = \begin{bmatrix}
                  1 & 0 & 1  \\
                  0 & 1 & -2 \\
                  0 & 0 & 1  \\
              \end{bmatrix}
    \end{aligned}
\end{equation*}

\subsection*{Task 2 [1 points]}
Using homogeneous coordinates:

\[
    p1 = \begin{bmatrix}
        1 \\
        1 \\
        1 \\
    \end{bmatrix}
\]

\[
    p = \begin{bmatrix}
        1.5 \\
        2.5 \\
        1   \\
    \end{bmatrix}
\]

\[
    u = p - p1 =
    u = \begin{bmatrix}
        0.5 \\
        1.5 \\
        0   \\
    \end{bmatrix}
\]

\[
    p\_90 = R90 * p =
    \begin{bmatrix}
        0 & -1 & 0 \\
        1 & 0  & 0 \\
        0 & 0  & 1 \\
    \end{bmatrix}
    *
    \begin{bmatrix}
        1.5 \\
        2.5 \\
        1   \\
    \end{bmatrix}
    =
    \begin{bmatrix}
        -2.5 \\
        1.5  \\
        1    \\
    \end{bmatrix}
\]


\[
    p1\_90 = R90 * p1 =
    \begin{bmatrix}
        0 & -1 & 0 \\
        1 & 0  & 0 \\
        0 & 0  & 1 \\
    \end{bmatrix}
    *
    \begin{bmatrix}
        1 \\
        1 \\
        1 \\
    \end{bmatrix}
    =
    \begin{bmatrix}
        -1 \\
        1  \\
        1  \\
    \end{bmatrix}
\]




\[
    u\_90 = R90 * u =
    \begin{bmatrix}
        0 & -1 & 0 \\
        1 & 0  & 0 \\
        0 & 0  & 1 \\
    \end{bmatrix}
    *
    \begin{bmatrix}
        0.5 \\
        1.5 \\
        0   \\
    \end{bmatrix}
    =
    \begin{bmatrix}
        -1.5 \\
        0.5  \\
        0    \\
    \end{bmatrix}
\]

Now for the translation:



\[
    T\_p\_90 = T * p\_90 =
    \begin{bmatrix}
        1 & 0 & 1  \\
        0 & 1 & -2 \\
        0 & 0 & 1  \\
    \end{bmatrix}
    *
    \begin{bmatrix}
        -2.5 \\
        1.5  \\
        1    \\
    \end{bmatrix}
    =
    \begin{bmatrix}
        -1.5 \\
        -0.5 \\
        1    \\
    \end{bmatrix}
\]



\[
    p' =
    \begin{bmatrix}
        -1.5 \\
        -0.5 \\
    \end{bmatrix}
\]


\[
    T\_p1\_90 = T * p1\_90 =
    \begin{bmatrix}
        1 & 0 & 1  \\
        0 & 1 & -2 \\
        0 & 0 & 1  \\
    \end{bmatrix}
    *
    \begin{bmatrix}
        -1 \\
        1  \\
        1  \\
    \end{bmatrix}
    =
    \begin{bmatrix}
        0  \\
        -1 \\
        1  \\
    \end{bmatrix}
\]



\[
    p1' =
    \begin{bmatrix}
        0  \\
        -1 \\
    \end{bmatrix}
\]



\[
    T\_u\_90 = T * u\_90 =
    \begin{bmatrix}
        1 & 0 & 1  \\
        0 & 1 & -2 \\
        0 & 0 & 1  \\
    \end{bmatrix}
    *
    \begin{bmatrix}
        -1.5 \\
        0.5  \\
        0    \\
    \end{bmatrix}
    =
    \begin{bmatrix}
        -1.5 \\
        0.5  \\
        0    \\
    \end{bmatrix}
\]


\[
    u' =
    \begin{bmatrix}
        -1.5 \\
        0.5  \\
    \end{bmatrix}
\]



We can see that T had no effect on u\_90, this is because u\_90 is a vector and so is not affected by translation.

\subsection*{Task 3 [1 points]}

\[
    S =
    \begin{bmatrix}
        2 & 0 & 0 \\
        0 & 2 & 0 \\
        0 & 0 & 1 \\
    \end{bmatrix}
\]



\[
    S\_p = S * p =
    \begin{bmatrix}
        2 & 0 & 0 \\
        0 & 2 & 0 \\
        0 & 0 & 1 \\
    \end{bmatrix}
    *
    \begin{bmatrix}
        -1.5 \\
        -0.5 \\
        1    \\
    \end{bmatrix}
    =
    \begin{bmatrix}
        -3 \\
        -1 \\
        1  \\
    \end{bmatrix}
\]
\[
    p'' =
    \begin{bmatrix}
        -3 \\
        -1 \\
        1  \\
    \end{bmatrix}
\]
\[
    S\_p1 = S * p1 =
    \begin{bmatrix}
        2 & 0 & 0 \\
        0 & 2 & 0 \\
        0 & 0 & 1 \\
    \end{bmatrix}
    *
    \begin{bmatrix}
        0  \\
        -1 \\
        0  \\
    \end{bmatrix}
    =
    \begin{bmatrix}
        0  \\
        -2 \\
        1  \\
    \end{bmatrix}
\]
\[
    p1'' =
    \begin{bmatrix}
        0  \\
        -2 \\
    \end{bmatrix}
\]
\[
    S\_u = S * u =
    \begin{bmatrix}
        2 & 0 & 0 \\
        0 & 2 & 0 \\
        0 & 0 & 1 \\
    \end{bmatrix}
    *
    \begin{bmatrix}
        -1.5 \\
        0.5  \\
        0    \\
    \end{bmatrix}
    =
    \begin{bmatrix}
        -3 \\
        1  \\
        0  \\
    \end{bmatrix}
\]
\[
    u'' =
    \begin{bmatrix}
        -3 \\
        1  \\
    \end{bmatrix}
\]

\subsection*{Task 4 [2 points]}
\[
    S^{-1} =
    \begin{bmatrix}
        1/2 & 0   & 0 \\
        0   & 1/2 & 0 \\
        0   & 0   & 1 \\
    \end{bmatrix}
\]


\[
    T^{-1} =
    \begin{bmatrix}
        1 & 0 & -1 \\
        0 & 1 & 2  \\
        0 & 0 & 1  \\
    \end{bmatrix}
\]


\[
    R\_90^{-1} =
    \begin{bmatrix}
        0  & 1 & 0 \\
        -1 & 0 & 0 \\
        0  & 0 & 1 \\
    \end{bmatrix}
\]


\[
    M = R\_90^{-1} * T^{-1} * S^{-1}
    \begin{bmatrix}
        0  & 1 & 0 \\
        -1 & 0 & 0 \\
        0  & 0 & 1 \\
    \end{bmatrix}
    *
    \begin{bmatrix}
        1 & 0 & -1 \\
        0 & 1 & 2  \\
        0 & 0 & 1  \\
    \end{bmatrix}
    *
    \begin{bmatrix}
        1/2 & 0   & 0 \\
        0   & 1/2 & 0 \\
        0   & 0   & 1 \\
    \end{bmatrix}
    =
    \begin{bmatrix}
        0    & 1/2 & 2 \\
        -1/2 & 0   & 1 \\
        0    & 0   & 1 \\
    \end{bmatrix}
\]



\[
    M * p'' =
    \begin{bmatrix}
        0    & 1/2 & 2 \\
        -1/2 & 0   & 1 \\
        0    & 0   & 1 \\
    \end{bmatrix}
    *
    \begin{bmatrix}
        -3 \\
        -1 \\
        1  \\
    \end{bmatrix}
    =
    \begin{bmatrix}
        1.5 \\
        2.5 \\
        1   \\
    \end{bmatrix}
    == p
\]



\[
    M * p1'' =
    \begin{bmatrix}
        0    & 1/2 & 2 \\
        -1/2 & 0   & 1 \\
        0    & 0   & 1 \\
    \end{bmatrix}
    *
    \begin{bmatrix}
        0  \\
        -2 \\
        1  \\
    \end{bmatrix}
    =
    \begin{bmatrix}
        1 \\
        1 \\
        1 \\
    \end{bmatrix}
    == p1
\]

\[
    M * u'' =
    \begin{bmatrix}
        0    & 1/2 & 2 \\
        -1/2 & 0   & 1 \\
        0    & 0   & 1 \\
    \end{bmatrix}
    *
    \begin{bmatrix}
        -3 \\
        1  \\
        0  \\
    \end{bmatrix}
    =
    \begin{bmatrix}
        0.5 \\
        1.5 \\
        0   \\
    \end{bmatrix}
    == u
\]



\cleardoublepage
\section*{Ex 2}

\[
    p1 = \begin{bmatrix}
        6 \\
        0 \\
        4 \\
    \end{bmatrix}
\]

\[
    p2 = \begin{bmatrix}
        2 \\
        0 \\
        0 \\
    \end{bmatrix}
\]

\[
    p3 = \begin{bmatrix}
        2 \\
        4 \\
        4 \\
    \end{bmatrix}
\]

\[
    p = \begin{bmatrix}
        4 \\
        1 \\
        3 \\
    \end{bmatrix}
\]

\[
    area\_of\_triangle = t = (p3 - p1) * (p2 - p1) =
    \begin{bmatrix}
        16  \\
        16  \\
        -16 \\
    \end{bmatrix}
\]

\[
    normal\_of\_sub\_triangle\_1 = t1 = (p3 - p) * (p1 - p) =
    \begin{bmatrix}
        4  \\
        4  \\
        -4 \\
    \end{bmatrix}
\]

\[
    normal\_of\_sub\_triangle\_2 = t2 = (p1 - p) * (p2 - p) =
    \begin{bmatrix}
        4  \\
        4  \\
        -4 \\
    \end{bmatrix}
\]


\[
    normal\_of\_sub\_triangle\_3 = t3 = (p2 - p) * (p3 - p) =
    \begin{bmatrix}
        8  \\
        8  \\
        -8 \\
    \end{bmatrix}
\]




We then compute the dot product between the sub triangles and t
\\
dot(t1, t) = 192 \\
dot(t2, t) = 192 \\
dot(t3, t) = 384 \\
Because all these dot products are positive, we can confirm that the point is indeed inside the triangle.




\cleardoublepage
\section*{Ex 3}



To demonstrate that the centroid of a triangle divides its medians in a 2:1 ratio using barycentric coordinates, let's rephrase the explanation:

Let's begin by visualizing a triangle with vertices $A$, $B$, and $C$, and we'll employ a barycentric coordinate system to analyze it. In this system, any point in the plane can be represented as $(\alpha, \beta, \gamma)$ with the constraint that $\alpha + \beta + \gamma$ equals 1.

Now, let's focus on the three medians of the triangle, denoted as $AM_a$, $BM_b$, and $CM_c$, where $M$ denotes the midpoint of the respective side. The barycentric coordinates of these medians can be expressed as follows:
\\
- For $AM_a$, we have $(\alpha, \beta, \gamma)$ with the condition that $\beta = \gamma$. \\
- Similarly, for $BM_b$, the barycentric coordinates are $(\alpha, \beta, \gamma)$ with $\alpha = \gamma$. \\
- And for $CM_c$, they are $(\alpha, \beta, \gamma)$ with $\alpha = \beta$. \\

Now, the centroid of the triangle, often represented as $G$, is the point where these medians intersect. We can summarize the conditions for $G$ as follows:
\\
- $\alpha + \beta + \gamma = 1$ (This is the general constraint for barycentric coordinates). \\
- $\beta = \gamma$ (from the $AM_a$ median). \\
- $\alpha = \gamma$ (from the $BM_b$ median). \\
- $\alpha = \beta$ (from the $CM_c$ median). \\

Solving this system of equations, we find that $\alpha = \beta = \gamma = 1/3$.

In other words, the barycentric coordinates for the centroid $G$ are $\left(1/3, 1/3, 1/3\right)$. This means that the centroid divides each of the medians in a 1:2 ratio (1 part to the centroid, 2 parts to the vertex). This can also be expressed as a 2:1 ratio (2 parts to the vertex, 1 part to the centroid), and it holds for any triangle, illustrating the desired result.





\cleardoublepage
\section*{Ex 4}
We can see that, when dealing with a given homogeneous coordinate $v = \begin{bmatrix} a \\ b \\ c \end{bmatrix}$, the corresponding Cartesian coordinate $f$ can be determined by dividing each component by the value of the third dimension and omitting that dimension. This operation yields $f = \begin{bmatrix} a/c \\ b/c \end{bmatrix}$.

We can see that for any point $p$, its transformed version can be represented as $p'' = \begin{bmatrix} 1 \\ y/x \end{bmatrix}$. This point transformation can be shown using the following matrix multiplication:

\[
    p'' = \begin{bmatrix} a & b & c \\ d & e & f \\ g & h & i \end{bmatrix} \cdot \begin{bmatrix} p_x \\ p_y \\ p_z \end{bmatrix} = \begin{bmatrix} 1 \\ p_y/p_x \\ 1 \end{bmatrix}
\]

To achieve $p''_x = 1$, we need to simplify $p_x$ by dividing it by itself. Converting homogeneous coordinates to Cartesian requires dividing all elements by the third element, we can conveniently set $p_z$ equal to $p_x$ to facilitate this simplification:

\[
    p'' = \begin{bmatrix} 1 & b & c \\ d & e & f \\ 1 & h & i \end{bmatrix} \cdot \begin{bmatrix} p_x \\ p_y \\ p_x \end{bmatrix} = \begin{bmatrix} p_x/p_x \\ 1/p_x \\ p_x / p_x \end{bmatrix} = \begin{bmatrix} 1 \\ 1 / p_x \\ 1 \end{bmatrix}
\]

To preserve the value of $y$, we set e = 1, as shown below:

\[
    p'' = \begin{bmatrix} 1 & b & c \\ d & 1 & f \\ 1 & h & i \end{bmatrix} \cdot \begin{bmatrix} p_x \\ p_y \\ p_x \end{bmatrix} = \begin{bmatrix} 1 \\ p_y / p_x \\ 1 \end{bmatrix}
\]

The transformation matrix can be represented as $ S = \begin{bmatrix} 1 & 0 & 0 \\ 0 & 1 & 0 \\ 1 & 0 & 0 \end{bmatrix} $ for all $p_x \in \mathbb{R}$, $p_x \neq 0$.

When a point $p$ lies on the y-axis (i.e., $p_x = 0$), there is no valid solution, as $p''_y$ would involve division by zero.



\cleardoublepage
\section*{Bonus exercise [2 points]}

\textbf{Consider the same task as in Exercise 4, but the line onto which the points are projected can be now arbitrary,
    and it is defined by a line equation $y = ax + b$, where $a, b \in \mathbb{R}$ are constants. Derive the matrix $M$ for this
    more general case.} \\

Represent the line equation as $l_1 : y = ax + b$ and put it in system with the previous exercise $4$ answer.

\begin{equation*}
    \begin{cases}
        l_1: y = \frac{p_y}{p_x} x \\
        y = ax + b
    \end{cases}
    \rightarrow
    \begin{cases}
        l_1: y = \frac{p_y}{p_x} x \\
        \frac{p_y}{p_x} x = ax + b
    \end{cases}
    \rightarrow
    \begin{cases}
        l_1: y = \frac{p_y}{p_x} x \\
        x (\frac{p_y - a p_x}{p_x}) = b
    \end{cases}
    \rightarrow
    \begin{cases}
        x = \frac{bp_x}{p_y - ap_x} \\
        y = \frac{bp_y}{p_y - ap_x}
    \end{cases}
\end{equation*}

Then, the matrix $M$ is given by:
\[
    M =
    \begin{bmatrix}
        M_{11} & M_{12} & M_{13} \\
        M_{21} & M_{22} & M_{23} \\
        M_{31} & M_{32} & M_{33}
    \end{bmatrix}
\]
where
\begin{equation*}
    p'' =
    \begin{cases}
        p_x'' = \frac{M_{11}p_x + M_{12}p_y + M_{13}p_x}{M_{31}p_x + M_{32}p_y + M_{33}p_x} = \frac{bp_x}{p_y - ap_x} \\\\
        p_y'' = \frac{M_{21}p_x + M_{22}p_y + M_{23}p_x}{M_{31}p_x + M_{32}p_y + M_{33}p_x} = \frac{bp_y}{p_y - ap_x}
    \end{cases}
\end{equation*}
so the only defined values in $M$ are:
\[
    M_{11} = b, M_{22} = b, M_{31} = -a, M_{32} = 1
\]
Finally:
\[
    M =
    \begin{bmatrix}
        b  & 0 & 0 \\
        0  & b & 0 \\
        -a & 1 & 0
    \end{bmatrix}
\]
\end{document}
